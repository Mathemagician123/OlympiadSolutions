\documentclass[11pt]{scrartcl}
\usepackage[utf8]{inputenc}
\usepackage{amsmath, amssymb, amsthm, bbm}
\usepackage{booktabs, verbatim, graphicx, framed}
\usepackage[sexy, hints]{evan}
\title{My Favorite Computational Problems}
\author{Anay Aggarwal}

\begin{document}

\maketitle
\section{Introduction}
This is a collection of some of my favorite computational problems. For many of these problems, I don't give full solutions. I give sketches with motivation, detailing all of the main steps of the problem. With 2023 contests in reference, the difficulty scale is as follows:
\begin{center}
  \begin{tabular}{|c|c|}
    0-2 & \text{Late AMC 10/12 Range} \\ \\
    3-4 & \text{AIME 8-11} \\ \\
    5-6 & \text{AIME 12-15} \\ \\
    7-8 & \text{HMMT February 8 - 9} \\ \\
    9-10 & \text{HMMT February 10 - Late OMO}
  \end{tabular}
\end{center}
Please note that the difficulty scale is very rough, and it is extremely subjective.
\newpage
\section{Hardness 0-2}
\begin{example}
  [By Manu Isaacs, for ARML PoTD]
  Let $A,B,C,D,E$ be complex numbers such that $ABCDE=-34+45i$ and $A+B+C+D+E=10+5i$. Additionally, when plotted on the complex plane, the five numbers form a regular pentagon. Compute the area of the circle that circumscribes this pentagon.
\end{example}
\begin{soln}
  We let the numbers be
$$A=(2+i)+r\omega_1,$$
$$B=(2+i)+r\omega_2,$$
$$C=(2+i)+r\omega_3,$$
$$D=(2+i)+r\omega_4,$$
$$E=(2+i)+r\omega_5.$$
Notice that $|r|$ is the radius of the circle circumscribing the pentagon. Here, $\omega_i$ are the 5th roots of unity. Using the above, we calculate $ABCDE$, keeping in mind that
$$0=\sum \omega_i = \sum \omega_i\omega_j = \sum \omega_i \omega_j \omega_k = \sum \omega_i \omega_j \omega_k \omega_l$$
and $\omega_1\omega_2\omega_3\omega_4\omega_5=1$:
$$-34+45i=ABCDE = ((2+i)+r\omega_1)((2+i)+r\omega_2)((2+i)+r\omega_3)((2+i)+r\omega_4)((2+i)+r\omega_5)$$
$$=(2+i)^5+r^5,$$
$$4+4i=r^5,$$
$$|r|=\sqrt{2},$$
$$\pi r^2 = \boxed{2\pi.}$$

\end{soln}
\begin{example}
   [2022 AIME I \#3]
   In isosceles trapezoid $ABCD$, parallel bases $\overline{AB}$ and $\overline{CD}$ have lengths $500$ and $650$, respectively, and $AD=BC=333$. The angle bisectors of $\angle{A}$ and $\angle{D}$ meet at $P$, and the angle bisectors of $\angle{B}$ and $\angle{C}$ meet at $Q$. Find $PQ$.
\end{example}
\begin{soln}
  I spent like 30 min on this and ended up coordinate bashing with the slope-tangent formula. The nice way is to note that the points on the angle bisector of $\angle A$ are equidistant from $AB$ and $AD$, and the points on the angle bisector of $\angle D$ are equidistant from $AD$ and $CD$, which implies that $P$ is on the midline, likewise with $Q$. Then let $AP$ hit $CD$ at $X$ and $BQ$ hit $CD$ at $Y$. Additionally,
$$\angle ADX=180-\angle DAB=180-2\angle DAX\implies \angle ADX=\angle DXA\implies DA=DX$$Thus, since the midline has length $575$, the answer is
$$575-2\left(\frac{333}{2}\right)=242$$
\end{soln}
\begin{example}
  [2021 ARML Individual Round (P7?)]
  A parallelogram with a vertex at $(0, 0)$ has its other three vertices as lattice points in the first quadrant. Given that the parallelogram has area $2021$, compute the maximum number of lattice points in the interior of the parallelogram.
\end{example}
\begin{soln}
  Pick's theorem.
\end{soln}
\newpage
\section{Hardness 3-4}
\begin{example}
  [AAIME 2023 P8 (\href{https://artofproblemsolving.com/community/c594864h2990894\_aaime\_released\_aamc\_year\_3\_\_aaime\_and\_}{Link})]
  Let $x,y,z\in\mathbb{C}$ be such that the following holds:
  $$xyz-4(x+y+z)=5$$
  $$xy+yz+xz=7$$
  $$(x^2+2)(y^2+2)(z^2+2)=246$$
  Find the sum of all possible values of $(x^2+6)(y^2+6)(z^2+6)$
\end{example}
\begin{soln}
  There are symmetric sums everywhere, except for that last equality. We need to figure out how to express this in terms
  of symmetric sums. Expanding out gives a term $x^2y^2+y^2z^2+x^2z^2$ that's tough to deal with. So how do we get around this?
  The idea is that $x^2+2=(x+2i)(x-2i)$ by the difference of squares. Now we're in business because the last equation becomes
  $$(x+2i)(y+2i)(z+2i)(x-2i)(y-2i)(z-2i)=246$$
  Now, instead of expanding out the first three terms and the second three terms, perhaps it is easier to just write the
  polynomial $P(t)$ with roots $x,y,z$, say $P(t)=(t-x)(t-y)(t-z)=t^3+at^2+bt+c$. By vieta's, the first two equations helps
  write $P(t)=t^3+at^2+7t+(4a-5)$. Then, the last condition gives $P(2i)P(-2i)=246$. This basically does it,
  plugging in $t=\pm 2i$, we solve for $a$. Then we can compute $(x^2+6)(y^2+6)(z^2+6)$ as $P(6i)P(-6i)$.
\end{soln}
\begin{example}
  [NIMO 2013 8.6, By Lewis Chen]
  Let $f(n)=\varphi(n^3)^{-1}$, where $\varphi(n)$ denotes the number of positive
\[ \frac{f(1)+f(3)+f(5)+\dots}{f(2)+f(4)+f(6)+\dots} = \frac{m}{n} \]
where $m$ and $n$ are relatively prime positive integers. Compute $100m+n$.

\end{example}
\begin{soln}
    Basically, we need to express the denominator in terms of the numerator. We can do this
  if we acknowledge that $\{2,4,6,\cdots,\}=\{2^k\cdot a|k\ge 1, a\equiv 1\pmod{2}\}$. So it's natural to compute $f(2^k\cdot a)$ for $a$ odd.
  Using multiplicativity of $\phi$, it's $\frac{f(a)}{2^{3k-1}}$. So for a fixed $a$, summing this over $k\ge 1$ we get $2/7$ be geometric series.
  The answer is then $7/2\implies 702$.

\end{soln}
\newpage
\section{Hardness 5-6}
\begin{example}
  [CARML P6, By Manu Isaacs (\href{https://artofproblemsolving.com/community/c594864h2492205}{Link})]
  Alpha draws a diagram. He first draws two mutually tangent circles $\omega_1, \omega_2$ with radius $1$, tangent to horizontal line $\ell$. He then draws a circle $\omega_3$ tangent to $\omega_1, \omega_2, $and $\ell$. Finally, he draws $\omega_4$ tangent to $\omega_3, \omega_2,$ and $\ell$. From then on, if $i$ is odd, $\omega_i$ is tangent to $\omega_{i-2}$, $\omega_1$, and $\ell$. If $i$ is even, $\omega_i$ is tangent to $\omega_{i-2}$, $\omega_2$, and $\ell$. Let $f(\omega)$ be the circumference of circle $\omega$. Find $\sum_{n=1}^{\infty}f(\omega_n)$.
\end{example}
\begin{soln}
  Let $a_i$ denote the radius of $\omega_i$. We have that $\frac{1}{\sqrt{a_3}}=2\implies a_3=\frac{1}{4}$ by Descartes circle theorem. In addition, $\frac{1}{\sqrt{a_4}}=3\implies a_4=\frac{1}{9}$. From then on, we split into two cases:
\newline
Case 1: $i$ is even. In this case, $\frac{1}{\sqrt{a_i}}=1+\frac{1}{\sqrt{a_{i-2}}}$. It's near trivial to verify with induction that $a_i=\frac{1}{\left(\frac{i}{2}+1\right)^2}$. This means that $\frac{1}{9}, \frac{1}{16}, ...$ occur exactly once here.
\newline
Case 2: $i$ is odd. The same phenomenon occurs here. We find that $\frac{1}{9}, \frac{1}{16}, ...$ occur exactly once here.
\newline
We have:
$$\sum_{n=1}^{\infty}f(\omega_n)=\sum_{n=1}^{\infty}2\pi a_n=2\pi \sum_{n=1}^{\infty}a_n$$
\newline
We know that in $a_i$, each $\frac{1}{k^2}$ (with $k\in \{1,2,...\}$) occurs twice except when $k=2$, in which it occurs only once. Then:
$$\sum_{n=1}^{\infty}a_n=2\left(\sum_{n=1}^{\infty}\frac{1}{n^2}\right)-\frac{1}{4}=\frac{\pi^2}{3}-\frac{1}{4}$$.
The answer is $2\pi \left(\frac{\pi^2}{3}-\frac{1}{4}\right)=2\pi\left(\frac{4\pi^2-3}{12}\right)=\frac{4\pi^3-3\pi}{6}$.
\end{soln}
\begin{example}
  [IOQM 2022-23 P20]
  For an integer $n\ge 3$ and a permutation $\sigma=(p_{1},p_{2},\cdots ,p_{n})$ of $\{1,2,\cdots , n\}$, we say $p_{l}$ is a $landmark$ point if $2\le l\le n-1$ and $(p_{l-1}-p_{l})(p_{l+1}-p_{l})>0$. For example , for $n=7$,
the permutation $(2,7,6,4,5,1,3)$ has four landmark points: $p_{2}=7$, $p_{4}=4$, $p_{5}=5$ and $p_{6}=1$. For a given $n\ge 3$ , let $L(n)$ denote the number of permutations of $\{1,2,\cdots ,n\}$ with exactly one landmark point. Find the maximum $n\ge 3$ for which $L(n)$ is a perfect square.
\end{example}
\begin{soln}
  The key observation is that if $p_l$ is a landmark then $p_{l-1}, p_l, p_{l+1}$ is either strictly increasing or decreasing. Let $\sigma$ be a permutation with exactly $1$ landmark point. Say the point is $p_k$. By the key observation, $p_k$ is either $n$ or $1$. WLOG it's $1$, multiply by $2$ in the end (the bijection is replacing each element $\varepsilon$ with $(n+1)-\varepsilon$, so this is allowed). Then there are $k-1$ elements on the left, and the rest are on the right of the $1$. There are $\binom{n-1}{k-1}$ ways to choose the elements on the left. Once you choose these they have to be sorted, and the elements on the right also have to be sorted, so $\binom{n-1}{k-1}$ is precisely the number of ways to put $1$ at $p_k$. Summing over $2\le k\le n-1$, there are $2^{n-1}-2$ ways. Multiplying by $2$, $L(n)=2^n-4$. The answer extraction is relatively easy, $2^n-4=4(2^{n-2}-1)$, so $2^{n-2}=k^2+1$ for some $k$. If $n\ge 4$, $k^2\equiv 3\pmod{4}$ which is bad. Note $L(3)=4$, so the answer is $3$.
\end{soln}
\begin{example}
  [2022 AIME I \#12]
  For any finite set $X$, let $|X|$ denote the number of elements in $X.$ Define$$S_n = \sum |A \cap B|,$$where the sum is taken over all ordered pairs $(A, B)$ such that $A$ and $B$ are subsets of $\{1, 2, 3, …, n\}$ with $|A| = |B|.$ For example, $S_2 = 4$ because the sum is taken over the pairs of subsets$$(A, B) \in \{ (\emptyset, \emptyset), (\{1\}, \{1\}), (\{1\}, \{2\}), (\{2\}, \{1\}), (\{2\}, \{2\}), (\{1, 2\}, \{1, 2\})\},$$giving $S_2 = 0 + 1 + 0 + 0 + 1 + 2 = 4.$ Let $\frac{S_{2022}}{S_{2021}} = \frac{p}{q},$ where $p$ and $q$ are relatively prime positive integers. Find the remainder when $p + q$ is divided by $1000.$
\end{example}
\begin{soln}
  Consider each element $x\in\{1,2,...,n\}$ individually. It's not hard to see that the number of ways to choose $A$ and $B$ such that $x\in A\cap B$ is
$$\sum_{k=0}^{n-1}\binom{n-1}{k}^2=\binom{2(n-1)}{n-1}$$By considering each possibility for the number of elements in $A$ and summing. So by double-counting
$$S_n=n\binom{2n-2}{n-1}$$And the rest is easy.
\end{soln}
\newpage
\section{Hardness 7-8}
\begin{example}
  [Own, for ARML PoTD]
  If $x_1, x_2, \cdots, x_{2022}$ are real numbers, find the maximum number of pairs $i,j$ with $|x_i-x_j|=2023$
\end{example}
\begin{soln}
  It's probably not too difficult to guess this problem's answer. Proving it is the hard part.
  Many problems with this sort of "connection" between numbers $i$ and $j$ involve graph theory.
  In fact, it is extremely useful to construct the graph with vertices $x_1, \cdots, x_{2022}$,
  with an edge between $x_i$ and $x_j$ if $|x_i-x_j|=2023$. The reason this works nicely is because the
  expression is reflexive, so it's not a directed graph. Once you have this graph, you can prove that
  it's bipartite relatively easily because it's easy to see that it has no odd cycle.
  Then we want the maximum number of edges in a bipartite graph, which is easy.
\end{soln}
\begin{example}
  [FidgetBoss\_4000 on AoPS]
  Let $N$ denote the number of ordered $n$-tuples $(a_1, a_2, …, a_{n})$ (where $n \geq 1$) that satisfy
$$\sum_{j=1}^{n} a_j = 2004$$and $a_i \in \{1, 2, …, 1001\}$ for all $1 \leq i\leq n$. Find the remainder when $N$ is divided by $1000$.
\end{example}
\begin{soln}
  It's pretty clearly a genfunc problem. Naturally, let $P(x)=x+x^2+\cdots+x^{1001}$. Let $[Q(x)]_n$ denote the $x^n$ coefficient of a polynomial $Q(x)$. We want $\sum_{n\ge 0}[P(x)^n]_{2004}=\left[\left(\sum_{n\ge 0} P(x)^n\right)\right]_{2004}$. After two applications of geometric series, we want the coefficient of $x^{2004}$ in $\frac{1-x}{x^{1002}-2x+1}$. Split this into two terms and apply the expansion of $\frac{1}{1+t}$ and the binomial theorem.
\end{soln}
\newpage
\section{Hardness 9-10}
\begin{example}
  [Vsamc on AoPS]
  Let $F$ be the number of functions $f: \{1, 2, \cdots, 200 \} \rightarrow \{1, 2, \cdots, 200 \}$ with $f(f(x)) = 2x$ for $1\leq x \leq 100$. Find the highest power of $2$ dividing $F$.
\end{example}
\begin{soln}
  There is a bijection to pairing up odd numbers between $1$ and $200$ (ordered pairs). First prove that $\nu_2(f(x))\le 1$ if $x$ is odd.
  The answer comes up to $100!/50!$ which has $\nu_2$ of $50$.
\end{soln}
\end{document}
