$\min(n)=\frac{r(r+1)}{2}$.
So $n=\frac{r(r+1)}{2}+kr$.

Claim: All powers of $2$ cannot be written this way.
Proof: $2^n=\frac{r(r+2k+1)}{2}\implies 2^i=r(r+2k+1)$. If $r$ is even, $r+2k+1$ is odd. If $r$ is odd, it immediately happens. Done.

Claim: All other numbers can be written this way.
Proof: So $2n=r(r+2k+1)$. If $r=2i$, we get $n=i(2(i+k)+1)$. A bit of parity work yields what we want. A similar argument works for odd $r$. Done.

So the answer is $100-|S|$ where $S=\{1,2,4,8,16,32,64\}\implies |S|=7$. This is just $\boxed{93}$.

Code verification:

#include <iostream>
#include <vector>

using namespace std;

int main()
{
  vector<int> list;
  for (int i=1; i<101; i++){
    list.push_back(i); //Filling list = {1,2,3,...,100}
  }
  for (int r=2; r<16; r++){
    for (int k=0; k<100; k++){ //Cycling through all possible r,k
      int num = r*(r+1)*0.5+r*k; // Writing the n for corresponding r,k
      for (int i=0; i<list.size(); i++){
        if (list[i]==num){
          list.erase(list.begin()+i); // If the number is in the list, erase it
        }
      }
    }
  }
  for (auto& entry: list){
    cout<<entry<<" "; // cout the list
  }
}


Returns the desired $\{1,2,4,8,16,32,64\}$.
